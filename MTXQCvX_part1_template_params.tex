\documentclass[9pt,]{article}
\usepackage[left=1in,top=1in,right=1in,bottom=1in]{geometry}
\newcommand*{\authorfont}{\fontfamily{phv}\selectfont}
\usepackage[]{mathpazo}


  \usepackage[T1]{fontenc}
  \usepackage[utf8]{inputenc}


\providecommand{\tightlist}{%
  \setlength{\itemsep}{0pt}\setlength{\parskip}{0pt}}

\usepackage{abstract}
\renewcommand{\abstractname}{}    % clear the title
\renewcommand{\absnamepos}{empty} % originally center

\renewenvironment{abstract}
 {{%
    \setlength{\leftmargin}{0mm}
    \setlength{\rightmargin}{\leftmargin}%
  }%
  \relax}
 {\endlist}

\makeatletter
\def\@maketitle{%
  \newpage
%  \null
%  \vskip 2em%
%  \begin{center}%
  \let \footnote \thanks
    {\fontsize{18}{20}\selectfont\raggedright  \setlength{\parindent}{0pt} \@title \par}%
}
%\fi
\makeatother




\setcounter{secnumdepth}{0}

\usepackage{color}
\usepackage{fancyvrb}
\newcommand{\VerbBar}{|}
\newcommand{\VERB}{\Verb[commandchars=\\\{\}]}
\DefineVerbatimEnvironment{Highlighting}{Verbatim}{commandchars=\\\{\}}
% Add ',fontsize=\small' for more characters per line
\usepackage{framed}
\definecolor{shadecolor}{RGB}{248,248,248}
\newenvironment{Shaded}{\begin{snugshade}}{\end{snugshade}}
\newcommand{\KeywordTok}[1]{\textcolor[rgb]{0.13,0.29,0.53}{\textbf{#1}}}
\newcommand{\DataTypeTok}[1]{\textcolor[rgb]{0.13,0.29,0.53}{#1}}
\newcommand{\DecValTok}[1]{\textcolor[rgb]{0.00,0.00,0.81}{#1}}
\newcommand{\BaseNTok}[1]{\textcolor[rgb]{0.00,0.00,0.81}{#1}}
\newcommand{\FloatTok}[1]{\textcolor[rgb]{0.00,0.00,0.81}{#1}}
\newcommand{\ConstantTok}[1]{\textcolor[rgb]{0.00,0.00,0.00}{#1}}
\newcommand{\CharTok}[1]{\textcolor[rgb]{0.31,0.60,0.02}{#1}}
\newcommand{\SpecialCharTok}[1]{\textcolor[rgb]{0.00,0.00,0.00}{#1}}
\newcommand{\StringTok}[1]{\textcolor[rgb]{0.31,0.60,0.02}{#1}}
\newcommand{\VerbatimStringTok}[1]{\textcolor[rgb]{0.31,0.60,0.02}{#1}}
\newcommand{\SpecialStringTok}[1]{\textcolor[rgb]{0.31,0.60,0.02}{#1}}
\newcommand{\ImportTok}[1]{#1}
\newcommand{\CommentTok}[1]{\textcolor[rgb]{0.56,0.35,0.01}{\textit{#1}}}
\newcommand{\DocumentationTok}[1]{\textcolor[rgb]{0.56,0.35,0.01}{\textbf{\textit{#1}}}}
\newcommand{\AnnotationTok}[1]{\textcolor[rgb]{0.56,0.35,0.01}{\textbf{\textit{#1}}}}
\newcommand{\CommentVarTok}[1]{\textcolor[rgb]{0.56,0.35,0.01}{\textbf{\textit{#1}}}}
\newcommand{\OtherTok}[1]{\textcolor[rgb]{0.56,0.35,0.01}{#1}}
\newcommand{\FunctionTok}[1]{\textcolor[rgb]{0.00,0.00,0.00}{#1}}
\newcommand{\VariableTok}[1]{\textcolor[rgb]{0.00,0.00,0.00}{#1}}
\newcommand{\ControlFlowTok}[1]{\textcolor[rgb]{0.13,0.29,0.53}{\textbf{#1}}}
\newcommand{\OperatorTok}[1]{\textcolor[rgb]{0.81,0.36,0.00}{\textbf{#1}}}
\newcommand{\BuiltInTok}[1]{#1}
\newcommand{\ExtensionTok}[1]{#1}
\newcommand{\PreprocessorTok}[1]{\textcolor[rgb]{0.56,0.35,0.01}{\textit{#1}}}
\newcommand{\AttributeTok}[1]{\textcolor[rgb]{0.77,0.63,0.00}{#1}}
\newcommand{\RegionMarkerTok}[1]{#1}
\newcommand{\InformationTok}[1]{\textcolor[rgb]{0.56,0.35,0.01}{\textbf{\textit{#1}}}}
\newcommand{\WarningTok}[1]{\textcolor[rgb]{0.56,0.35,0.01}{\textbf{\textit{#1}}}}
\newcommand{\AlertTok}[1]{\textcolor[rgb]{0.94,0.16,0.16}{#1}}
\newcommand{\ErrorTok}[1]{\textcolor[rgb]{0.64,0.00,0.00}{\textbf{#1}}}
\newcommand{\NormalTok}[1]{#1}
\usepackage{longtable,booktabs}


\usepackage{graphicx}


\title{MTXQCvX - Part1: pSIRM \thanks{Kempa Lab - Template MTXQCvX part1 - processed `September 25, 2018'}  }



\author{\Large test\vspace{0.05in} \newline\normalsize\emph{test}  }


\date{}

\usepackage{titlesec}

\titleformat*{\section}{\normalsize\bfseries}
\titleformat*{\subsection}{\normalsize\itshape}
\titleformat*{\subsubsection}{\normalsize\itshape}
\titleformat*{\paragraph}{\normalsize\itshape}
\titleformat*{\subparagraph}{\normalsize\itshape}


\usepackage{natbib}
\bibliographystyle{apsr}



\newtheorem{hypothesis}{Hypothesis}
\usepackage{setspace}

\makeatletter
\@ifpackageloaded{hyperref}{}{%
\ifxetex
  \usepackage[setpagesize=false, % page size defined by xetex
              unicode=false, % unicode breaks when used with xetex
              xetex]{hyperref}
\else
  \usepackage[unicode=true]{hyperref}
\fi
}
\@ifpackageloaded{color}{
    \PassOptionsToPackage{usenames,dvipsnames}{color}
}{%
    \usepackage[usenames,dvipsnames]{color}
}
\makeatother
\hypersetup{breaklinks=true,
            bookmarks=true,
            pdfauthor={test (test)},
             pdfkeywords = {MTXQCvX, GC-MS, metabolomics, data analysis and processing},  
            pdftitle={MTXQCvX - Part1: pSIRM},
            colorlinks=true,
            citecolor=blue,
            urlcolor=blue,
            linkcolor=magenta,
            pdfborder={0 0 0}}
\urlstyle{same}  % don't use monospace font for urls



\begin{document}
	
% \pagenumbering{arabic}% resets `page` counter to 1 
%
% \maketitle

{% \usefont{T1}{pnc}{m}{n}
\setlength{\parindent}{0pt}
\thispagestyle{plain}
{\fontsize{18}{20}\selectfont\raggedright 
\maketitle  % title \par  

}

{
   \vskip 13.5pt\relax \normalsize\fontsize{11}{12} 
\textbf{\authorfont test} \hskip 15pt \emph{\small test}   

}

}



{
\hypersetup{linkcolor=black}
\setcounter{tocdepth}{2}
\tableofcontents
}




\begin{abstract}

    \hbox{\vrule height .2pt width 39.14pc}

    \vskip 8.5pt % \small 

\noindent This document provides an evaluation of GC-MS derived metabolomics data.
It asseses GC-MS performance, the absolute quantification and the stable
isotope incorporation. ADD HERE FURTHER PROJECT RELEVANT FACTS.


\vskip 8.5pt \noindent \emph{Keywords}: MTXQCvX, GC-MS, metabolomics, data analysis and processing \par

    \hbox{\vrule height .2pt width 39.14pc}



\end{abstract}


\vskip 6.5pt

\noindent  \section{MTXQCvX part1}\label{mtxqcvx-part1}

\subsection{Summary}\label{summary}

** Summarise your major findings and important details. DO NOT skip this
part.**

\subsection{General project settings}\label{general-project-settings}

\begin{verbatim}
## 
## Attaching package: 'gplots'
\end{verbatim}

\begin{verbatim}
## The following object is masked from 'package:stats':
## 
##     lowess
\end{verbatim}

\subsection{Data import}\label{data-import}

\begin{verbatim}
## MTXQCparams.csv imported!
\end{verbatim}

\begin{verbatim}
## Maui_params.csv imported.
\end{verbatim}

\begin{verbatim}
## Required table containing additional Quant1-values detected!
\end{verbatim}

\begin{verbatim}
## File imported! Annotation_allbatches.csv
\end{verbatim}

\begin{verbatim}
## File imported! Sample_extract_allbatches.csv
\end{verbatim}

\begin{verbatim}
## File imported! InternalStandard.csv
\end{verbatim}

\begin{verbatim}
## File imported! Alcane_intensities.csv
\end{verbatim}

\begin{verbatim}
## File imported! MassSum-73.csv
\end{verbatim}

\begin{verbatim}
## File imported! PeakDensities-Chroma.csv
\end{verbatim}

\begin{verbatim}
## File imported! quantMassAreasMatrix_manVal.csv
\end{verbatim}

\begin{verbatim}
## File imported! pSIRM_SpectraData.csv
\end{verbatim}

\begin{verbatim}
## File imported! DataMatrix.csv
\end{verbatim}

\begin{verbatim}
## Correct column names in file sample_extracts.csv
\end{verbatim}

\begin{verbatim}
## Correct column names in sample annotation
\end{verbatim}

\begin{verbatim}
## Input files checked!
\end{verbatim}

\begin{verbatim}
## Annotation and Sample_extract.csv correctly imported!
\end{verbatim}

\section{MTXQC - GC-MS perfomance}\label{mtxqc---gc-ms-perfomance}

\subsection{Alkane standards}\label{alkane-standards}

\begin{verbatim}
## QC-metric succesfully exported: alkanes
\end{verbatim}

\begin{figure}

{\centering \includegraphics{test/figure/MTXQCp1-alkanes-1} 

}

\caption{Alkane intensities summarised per each file. Drop of intensities shows questionable files.}\label{fig:alkanes}
\end{figure}

\subsection{Data normalization}\label{data-normalization}

\subsubsection{Internal standard cinnamic
acid}\label{internal-standard-cinnamic-acid}

\begin{verbatim}
## QC-metric succesfully exported: cinacid
\end{verbatim}

\begin{figure}

{\centering \includegraphics{test/figure/MTXQCp1-cinacid_plot-1} 

}

\caption{Quantification of internal extraction standard}\label{fig:cinacid_plot}
\end{figure}

\subsubsection{Sum of Area of annotated metabolites per
file}\label{sum-of-area-of-annotated-metabolites-per-file}

\begin{verbatim}
## Files with less than 50% of max(N) should be excluded from SumofArea normalisation.
\end{verbatim}

\begin{verbatim}
## QC-metric succesfully exported: sumofarea
\end{verbatim}

\begin{longtable}[]{@{}lr@{}}
\toprule
Batch\_Id & n\_50\tabularnewline
\midrule
\endhead
e18057cz & 39.0\tabularnewline
e18060cz & 44.0\tabularnewline
e18061cz & 37.5\tabularnewline
\bottomrule
\end{longtable}

\begin{figure}

{\centering \includegraphics{test/figure/MTXQCp1-nb_annotation-1} 

}

\caption{Count N: Annotated intermediates per file. Evaluate careful for SumOfArea normalisation.}\label{fig:nb_annotation}
\end{figure}

\begin{figure}

{\centering \includegraphics{test/figure/MTXQCp1-SumofArea-1} 

}

\caption{Total peak area of all annotated metabolite per file.}\label{fig:SumofArea}
\end{figure}

\subsection{Derivatization check}\label{derivatization-check}

\begin{verbatim}
## QC-metric succesfully exported: mz73
\end{verbatim}

\subsection{HeatMap - GC-MS
performance}\label{heatmap---gc-ms-performance}

\begin{longtable}[]{@{}lrl@{}}
\caption{Summary of parameter evaluating GC-Performance}\tabularnewline
\toprule
Batch\_Id & qc\_metric & title\tabularnewline
\midrule
\endfirsthead
\toprule
Batch\_Id & qc\_metric & title\tabularnewline
\midrule
\endhead
e18057cz & 0.9371664 & alkanes\tabularnewline
e18060cz & 0.9104125 & alkanes\tabularnewline
e18061cz & 0.9173959 & alkanes\tabularnewline
e18057cz & 0.3914688 & cinacid\tabularnewline
e18060cz & 0.6531375 & cinacid\tabularnewline
e18061cz & 0.6563009 & cinacid\tabularnewline
e18057cz & 0.6818946 & mz73\tabularnewline
e18060cz & 0.8585613 & mz73\tabularnewline
e18061cz & 0.7720460 & mz73\tabularnewline
e18057cz & 0.6512975 & sumofarea\tabularnewline
e18060cz & 0.7576846 & sumofarea\tabularnewline
e18061cz & 0.6644690 & sumofarea\tabularnewline
\bottomrule
\end{longtable}

\begin{verbatim}
## Export of GC-Performance values done!
\end{verbatim}

\section{MTXQC - Quantitative
metabolomics}\label{mtxqc---quantitative-metabolomics}

\subsection{Generation of ManualQuantTable: Quant-Standards
(Qstd)}\label{generation-of-manualquanttable-quant-standards-qstd}

\begin{verbatim}
## File imported! quant1_values.csv
\end{verbatim}

\begin{verbatim}
## Correct matching of ManualQuantTable files and annotation file content!
\end{verbatim}

\begin{verbatim}
## ManualQuantTable for standard calibration curves has been generated. Quant1_v3
\end{verbatim}

\begin{verbatim}
## ManualQuantTable generated and exported!
\end{verbatim}

\subsection{Generation of ManualQuantTable: Additional calibration
curves
(Qadd)}\label{generation-of-manualquanttable-additional-calibration-curves-qadd}

\begin{verbatim}
## Additional quant1-values imported for metabolites:  3
\end{verbatim}

\begin{verbatim}
## Additional calibration curves have been defined for all included batches!
\end{verbatim}

\begin{verbatim}
## Additional calibration curves have been duplicated and added for all batches!
\end{verbatim}

\begin{verbatim}
## ManualQuantTable for additional calibration curves has been generated. Quant1-values: Quant_ext
\end{verbatim}

\begin{figure}

{\centering \includegraphics{test/figure/MTXQCp1-add_calcurves-1} 

}

\caption{Additional Calibration curves}\label{fig:add_calcurves1}
\end{figure}\begin{figure}

{\centering \includegraphics{test/figure/MTXQCp1-add_calcurves-2} 

}

\caption{Additional Calibration curves}\label{fig:add_calcurves2}
\end{figure}\begin{figure}

{\centering \includegraphics{test/figure/MTXQCp1-add_calcurves-3} 

}

\caption{Additional Calibration curves}\label{fig:add_calcurves3}
\end{figure}

\begin{verbatim}
## Additional Quant-Standards have been added to MQT_integrated.csv
\end{verbatim}

\subsection{Determination of calibration
curves}\label{determination-of-calibration-curves}

\begin{verbatim}
## top5_QMQcurveInfo.csv generated!
\end{verbatim}

\begin{Shaded}
\begin{Highlighting}[]
\ControlFlowTok{if}\NormalTok{ (}\KeywordTok{nrow}\NormalTok{(qc_calcurve }\OperatorTok{!=}\StringTok{ }\DecValTok{0}\NormalTok{)) \{}
  \KeywordTok{ggplot}\NormalTok{(qc_calcurve, }\KeywordTok{aes}\NormalTok{(Lettercode, Par_value, }\DataTypeTok{color =}\NormalTok{ Parameter)) }\OperatorTok{+}
\StringTok{        }\KeywordTok{geom_point}\NormalTok{(}\KeywordTok{aes}\NormalTok{(}\DataTypeTok{shape =}\NormalTok{ Parameter), }\DataTypeTok{size =} \DecValTok{3}\NormalTok{) }\OperatorTok{+}
\StringTok{    }\KeywordTok{coord_flip}\NormalTok{() }\OperatorTok{+}
\StringTok{    }\KeywordTok{ggtitle}\NormalTok{(}\StringTok{'Calibration curve: adj. R square and nb of data points'}\NormalTok{) }\OperatorTok{+}
\StringTok{    }\KeywordTok{ylim}\NormalTok{(}\DecValTok{0}\NormalTok{,}\DecValTok{1}\NormalTok{) }\OperatorTok{+}
\StringTok{    }\KeywordTok{geom_hline}\NormalTok{(}\KeywordTok{aes}\NormalTok{(}\DataTypeTok{yintercept =} \FloatTok{0.75}\NormalTok{), }\DataTypeTok{linetype =} \StringTok{'dashed'}\NormalTok{, }\DataTypeTok{color =} \StringTok{'grey30'}\NormalTok{) }\OperatorTok{+}
\StringTok{    }\KeywordTok{scale_color_manual}\NormalTok{(}\DataTypeTok{values =} \KeywordTok{c}\NormalTok{(}\StringTok{'tomato3'}\NormalTok{,}\StringTok{'black'}\NormalTok{)) }\OperatorTok{+}
\StringTok{    }\KeywordTok{scale_shape_manual}\NormalTok{(}\DataTypeTok{values =} \KeywordTok{c}\NormalTok{(}\DecValTok{17}\NormalTok{,}\DecValTok{20}\NormalTok{)) }\OperatorTok{+}
\StringTok{    }\KeywordTok{facet_grid}\NormalTok{(Origin }\OperatorTok{~}\StringTok{ }\NormalTok{Batch_Id, }\DataTypeTok{scales =} \StringTok{"free_y"}\NormalTok{) }\OperatorTok{+}
\StringTok{        }\KeywordTok{xlab}\NormalTok{(}\StringTok{'Derivate'}\NormalTok{) }\OperatorTok{+}
\StringTok{        }\KeywordTok{ylab}\NormalTok{(}\StringTok{'Parameter value in (-)'}\NormalTok{) }\OperatorTok{+}
\StringTok{    }\KeywordTok{theme}\NormalTok{(}\DataTypeTok{legend.position =} \StringTok{"bottom"}\NormalTok{)}
\NormalTok{\}}
\end{Highlighting}
\end{Shaded}

\begin{figure}

{\centering \includegraphics{test/figure/MTXQCp1-calcurve_par-1} 

}

\caption{Calibration curves: Nb. of data points.}\label{fig:calcurve_par}
\end{figure}

\begin{figure}

{\centering \includegraphics{test/figure/MTXQCp1-calcurve_linrange-1} 

}

\caption{Limits of quantifiable range per metabolite}\label{fig:calcurve_linrange}
\end{figure}

\subsection{Evaluation of experimental
data}\label{evaluation-of-experimental-data}

\subsubsection{Determination extraction
factor}\label{determination-extraction-factor}

\begin{verbatim}
## The quantification factor for that experimental setup: 0.333333333333333
\end{verbatim}

\begin{verbatim}
## The sample factor for that experimental setup: 1
\end{verbatim}

\begin{verbatim}
## The extraction factor for that experimental setup: 0.333333333333333
\end{verbatim}

\subsubsection{Quantification range and
limits}\label{quantification-range-and-limits}

\begin{verbatim}
## Position of data points regarding calibration curves evaluated.
\end{verbatim}

\begin{figure}

{\centering \includegraphics{test/figure/MTXQCp1-samples_range-1} 

}

\caption{Distribution of data points regarding linear range of the calibration curve}\label{fig:samples_range1}
\end{figure}\begin{figure}

{\centering \includegraphics{test/figure/MTXQCp1-samples_range-2} 

}

\caption{Distribution of data points regarding linear range of the calibration curve}\label{fig:samples_range2}
\end{figure}

\subsubsection{Absolute quantification
samples}\label{absolute-quantification-samples}

\begin{center}\includegraphics{test/figure/MTXQCp1-calcurve_sample-1} \end{center}

\begin{center}\includegraphics{test/figure/MTXQCp1-calcurve_sample-2} \end{center}

\begin{center}\includegraphics{test/figure/MTXQCp1-calcurve_sample-3} \end{center}

\subsubsection{Normalisation of absolute
quantities}\label{normalisation-of-absolute-quantities}

\begin{verbatim}
## Absolute quantification and normalisation have been performed: CalculationFileData.csv
\end{verbatim}

\subsection{HeatMap - Quantification}\label{heatmap---quantification}

\clearpage

\section{MTXQC - Stable isotope
incorporation}\label{mtxqc---stable-isotope-incorporation}

\subsection{NA count}\label{na-count}

\begin{figure}

{\centering \includegraphics{test/figure/MTXQCp1-SE_score-1} 

}

\caption{Missing values in mass isotopomer distributions (MID).}\label{fig:SE_score}
\end{figure}

\subsection{3-Lowest of MID}\label{lowest-of-mid}

\subsection{3-Lowest of MID}\label{lowest-of-mid-1}

\begin{figure}

{\centering \includegraphics{test/figure/MTXQCp1-MID_quality-1} 

}

\caption{MID quality}\label{fig:MID_quality}
\end{figure}

\subsection{\texorpdfstring{\(^{13}C\)-Isotope
incorporation}{\^{}\{13\}C-Isotope incorporation}}\label{c-isotope-incorporation}

\begin{verbatim}
## No data for t=0 in the experimental setup defined!
\end{verbatim}

\clearpage
\#\# Heatmap Isotope incorporation

\section{MTXQC Heatmap compilation: Quantifitation and stable isotope
incorporation}\label{mtxqc-heatmap-compilation-quantifitation-and-stable-isotope-incorporation}

\begin{figure}

{\centering \includegraphics{test/figure/MTXQCp1-heatmaps-1} 

}

\caption{MTXQCvX - Heatmap overview}\label{fig:heatmaps}
\end{figure}

\emph{End of the document}
\newpage
\singlespacing 
\end{document}