\documentclass[9pt,]{article}
\usepackage[left=1in,top=1in,right=1in,bottom=1in]{geometry}
\newcommand*{\authorfont}{\fontfamily{phv}\selectfont}
\usepackage[]{mathpazo}


  \usepackage[T1]{fontenc}
  \usepackage[utf8]{inputenc}


\providecommand{\tightlist}{%
  \setlength{\itemsep}{0pt}\setlength{\parskip}{0pt}}

\usepackage{abstract}
\renewcommand{\abstractname}{}    % clear the title
\renewcommand{\absnamepos}{empty} % originally center

\renewenvironment{abstract}
 {{%
    \setlength{\leftmargin}{0mm}
    \setlength{\rightmargin}{\leftmargin}%
  }%
  \relax}
 {\endlist}

\makeatletter
\def\@maketitle{%
  \newpage
%  \null
%  \vskip 2em%
%  \begin{center}%
  \let \footnote \thanks
    {\fontsize{18}{20}\selectfont\raggedright  \setlength{\parindent}{0pt} \@title \par}%
}
%\fi
\makeatother




\setcounter{secnumdepth}{0}

\usepackage{color}
\usepackage{fancyvrb}
\newcommand{\VerbBar}{|}
\newcommand{\VERB}{\Verb[commandchars=\\\{\}]}
\DefineVerbatimEnvironment{Highlighting}{Verbatim}{commandchars=\\\{\}}
% Add ',fontsize=\small' for more characters per line
\usepackage{framed}
\definecolor{shadecolor}{RGB}{248,248,248}
\newenvironment{Shaded}{\begin{snugshade}}{\end{snugshade}}
\newcommand{\KeywordTok}[1]{\textcolor[rgb]{0.13,0.29,0.53}{\textbf{#1}}}
\newcommand{\DataTypeTok}[1]{\textcolor[rgb]{0.13,0.29,0.53}{#1}}
\newcommand{\DecValTok}[1]{\textcolor[rgb]{0.00,0.00,0.81}{#1}}
\newcommand{\BaseNTok}[1]{\textcolor[rgb]{0.00,0.00,0.81}{#1}}
\newcommand{\FloatTok}[1]{\textcolor[rgb]{0.00,0.00,0.81}{#1}}
\newcommand{\ConstantTok}[1]{\textcolor[rgb]{0.00,0.00,0.00}{#1}}
\newcommand{\CharTok}[1]{\textcolor[rgb]{0.31,0.60,0.02}{#1}}
\newcommand{\SpecialCharTok}[1]{\textcolor[rgb]{0.00,0.00,0.00}{#1}}
\newcommand{\StringTok}[1]{\textcolor[rgb]{0.31,0.60,0.02}{#1}}
\newcommand{\VerbatimStringTok}[1]{\textcolor[rgb]{0.31,0.60,0.02}{#1}}
\newcommand{\SpecialStringTok}[1]{\textcolor[rgb]{0.31,0.60,0.02}{#1}}
\newcommand{\ImportTok}[1]{#1}
\newcommand{\CommentTok}[1]{\textcolor[rgb]{0.56,0.35,0.01}{\textit{#1}}}
\newcommand{\DocumentationTok}[1]{\textcolor[rgb]{0.56,0.35,0.01}{\textbf{\textit{#1}}}}
\newcommand{\AnnotationTok}[1]{\textcolor[rgb]{0.56,0.35,0.01}{\textbf{\textit{#1}}}}
\newcommand{\CommentVarTok}[1]{\textcolor[rgb]{0.56,0.35,0.01}{\textbf{\textit{#1}}}}
\newcommand{\OtherTok}[1]{\textcolor[rgb]{0.56,0.35,0.01}{#1}}
\newcommand{\FunctionTok}[1]{\textcolor[rgb]{0.00,0.00,0.00}{#1}}
\newcommand{\VariableTok}[1]{\textcolor[rgb]{0.00,0.00,0.00}{#1}}
\newcommand{\ControlFlowTok}[1]{\textcolor[rgb]{0.13,0.29,0.53}{\textbf{#1}}}
\newcommand{\OperatorTok}[1]{\textcolor[rgb]{0.81,0.36,0.00}{\textbf{#1}}}
\newcommand{\BuiltInTok}[1]{#1}
\newcommand{\ExtensionTok}[1]{#1}
\newcommand{\PreprocessorTok}[1]{\textcolor[rgb]{0.56,0.35,0.01}{\textit{#1}}}
\newcommand{\AttributeTok}[1]{\textcolor[rgb]{0.77,0.63,0.00}{#1}}
\newcommand{\RegionMarkerTok}[1]{#1}
\newcommand{\InformationTok}[1]{\textcolor[rgb]{0.56,0.35,0.01}{\textbf{\textit{#1}}}}
\newcommand{\WarningTok}[1]{\textcolor[rgb]{0.56,0.35,0.01}{\textbf{\textit{#1}}}}
\newcommand{\AlertTok}[1]{\textcolor[rgb]{0.94,0.16,0.16}{#1}}
\newcommand{\ErrorTok}[1]{\textcolor[rgb]{0.64,0.00,0.00}{\textbf{#1}}}
\newcommand{\NormalTok}[1]{#1}


\usepackage{graphicx}


\title{MTXQCvX - Part3: ManualValidation template \thanks{Template MTXQCvX part3 written by Christin Zasada, Kempa Lab}  }



\author{\Large NAME\vspace{0.05in} \newline\normalsize\emph{LABNAME}   \and \Large NAME\vspace{0.05in} \newline\normalsize\emph{LABNAME}  }


\date{}

\usepackage{titlesec}

\titleformat*{\section}{\normalsize\bfseries}
\titleformat*{\subsection}{\normalsize\itshape}
\titleformat*{\subsubsection}{\normalsize\itshape}
\titleformat*{\paragraph}{\normalsize\itshape}
\titleformat*{\subparagraph}{\normalsize\itshape}


\usepackage{natbib}
\bibliographystyle{apsr}



\newtheorem{hypothesis}{Hypothesis}
\usepackage{setspace}

\makeatletter
\@ifpackageloaded{hyperref}{}{%
\ifxetex
  \usepackage[setpagesize=false, % page size defined by xetex
              unicode=false, % unicode breaks when used with xetex
              xetex]{hyperref}
\else
  \usepackage[unicode=true]{hyperref}
\fi
}
\@ifpackageloaded{color}{
    \PassOptionsToPackage{usenames,dvipsnames}{color}
}{%
    \usepackage[usenames,dvipsnames]{color}
}
\makeatother
\hypersetup{breaklinks=true,
            bookmarks=true,
            pdfauthor={NAME (LABNAME) and NAME (LABNAME)},
             pdfkeywords = {MTXQCvX, manual validation},  
            pdftitle={MTXQCvX - Part3: ManualValidation template},
            colorlinks=true,
            citecolor=blue,
            urlcolor=blue,
            linkcolor=magenta,
            pdfborder={0 0 0}}
\urlstyle{same}  % don't use monospace font for urls



\begin{document}
	
% \pagenumbering{arabic}% resets `page` counter to 1 
%
% \maketitle

{% \usefont{T1}{pnc}{m}{n}
\setlength{\parindent}{0pt}
\thispagestyle{plain}
{\fontsize{18}{20}\selectfont\raggedright 
\maketitle  % title \par  

}

{
   \vskip 13.5pt\relax \normalsize\fontsize{11}{12} 
\textbf{\authorfont NAME} \hskip 15pt \emph{\small LABNAME}   \par \textbf{\authorfont NAME} \hskip 15pt \emph{\small LABNAME}   

}

}



{
\hypersetup{linkcolor=black}
\setcounter{tocdepth}{2}
\tableofcontents
}




\begin{abstract}

    \hbox{\vrule height .2pt width 39.14pc}

    \vskip 8.5pt % \small 

\noindent This document provides the manual validation of GC-MS derived data
(MTXQC part 3). Thats has been processed through MTXQCvX part1 before.
It transforms your MAUI exports into easily modifiable tables (PrepData)
and re-transform them after manual validation into csv-files usable for
another round of MTXQC (EvalQuant; EvalInc). In case of Metmax input
files run MTXQC part 4 first.


\vskip 8.5pt \noindent \emph{Keywords}: MTXQCvX, manual validation \par

    \hbox{\vrule height .2pt width 39.14pc}



\end{abstract}


\vskip 6.5pt

\noindent  \section{Data transformation for convenient manual
validation}\label{data-transformation-for-convenient-manual-validation}

\begin{Shaded}
\begin{Highlighting}[]
\CommentTok{#if (params$inputformat != "maui") \{}
 \CommentTok{# message("This input format is currently not integrated in this module! Sorry!")}
\CommentTok{#  knitr::knit_exit()}
\CommentTok{#\}}

\ControlFlowTok{if}\NormalTok{ ((params}\OperatorTok{$}\NormalTok{prep }\OperatorTok{!=}\StringTok{ "none"}\NormalTok{) }\OperatorTok{&}\StringTok{ }\NormalTok{(params}\OperatorTok{$}\NormalTok{eval }\OperatorTok{!=}\StringTok{ "none"}\NormalTok{)) \{}
  \KeywordTok{message}\NormalTok{(}\StringTok{"Please select only one action at a time - eiter data transformation or data integration!"}\NormalTok{)}
\NormalTok{  knitr}\OperatorTok{::}\KeywordTok{knit_exit}\NormalTok{()}
\NormalTok{\}}
\end{Highlighting}
\end{Shaded}

\begin{Shaded}
\begin{Highlighting}[]
\CommentTok{#MOD!}
\NormalTok{set_input =}\StringTok{ "input/"}
\NormalTok{set_output =}\StringTok{ "output/"}

\NormalTok{## subfolder for postprocessing}
\NormalTok{set_val =}\StringTok{ }\KeywordTok{paste0}\NormalTok{(params}\OperatorTok{$}\NormalTok{folder, }\StringTok{"/"}\NormalTok{)}

\ControlFlowTok{if}\NormalTok{ (set_val }\OperatorTok{==}\StringTok{ ""}\NormalTok{) \{}
  \KeywordTok{message}\NormalTok{(}\StringTok{"Please define a folder!"}\NormalTok{)}
\NormalTok{  knitr}\OperatorTok{::}\KeywordTok{knit_exit}\NormalTok{()}
\NormalTok{\}}

\CommentTok{#directory definition and figure_name definition}
\ControlFlowTok{if}\NormalTok{ (params}\OperatorTok{$}\NormalTok{spath }\OperatorTok{==}\StringTok{ ""}\NormalTok{) \{}
\NormalTok{  path_setup =}\StringTok{ ""}
\NormalTok{  set_fig =}\StringTok{ }\KeywordTok{paste0}\NormalTok{(path_setup, }\StringTok{'figure/MTXQCp3-'}\NormalTok{)}
\NormalTok{\} }\ControlFlowTok{else}\NormalTok{ \{}
\NormalTok{  path_setup =}\StringTok{ }\KeywordTok{paste0}\NormalTok{(params}\OperatorTok{$}\NormalTok{spath, }\StringTok{"/"}\NormalTok{)}
\NormalTok{  set_fig =}\StringTok{ }\KeywordTok{paste0}\NormalTok{(path_setup, }\StringTok{'figure/MTXQCp3-'}\NormalTok{)}
\NormalTok{\}}

\NormalTok{knitr}\OperatorTok{::}\NormalTok{opts_chunk}\OperatorTok{$}\KeywordTok{set}\NormalTok{(}\DataTypeTok{fig.width =} \DecValTok{5}\NormalTok{, }\DataTypeTok{fig.align =} \StringTok{'center'}\NormalTok{, }\DataTypeTok{fig.height =} \DecValTok{4}\NormalTok{,}
                      \DataTypeTok{fig.path =}\NormalTok{ set_fig,  }
                      \DataTypeTok{echo =} \OtherTok{FALSE}\NormalTok{,  }\CommentTok{#TRUE - show R code}
                      \DataTypeTok{warning =} \OtherTok{FALSE}\NormalTok{, }\CommentTok{#show warnings}
                      \DataTypeTok{message =} \OtherTok{TRUE}\NormalTok{) }\CommentTok{#show messages}

\CommentTok{#Create a folder and stop processing if it is already present performing PrepData}
\ControlFlowTok{if}\NormalTok{ (params}\OperatorTok{$}\NormalTok{prep }\OperatorTok{!=}\StringTok{ "none"}\NormalTok{) \{}
  \ControlFlowTok{if}\NormalTok{ (}\OperatorTok{!}\KeywordTok{dir.exists}\NormalTok{(}\KeywordTok{file.path}\NormalTok{(}\KeywordTok{paste0}\NormalTok{(path_setup, set_output, set_val)))) \{}
      \KeywordTok{dir.create}\NormalTok{(}\KeywordTok{paste0}\NormalTok{(path_setup, set_output, set_val))}
\NormalTok{  \} }\ControlFlowTok{else}\NormalTok{ \{}
     \KeywordTok{message}\NormalTok{(}\StringTok{"Folder already exists! Please define a new folder where to save transformed data."}\NormalTok{)}
\NormalTok{  \}}
\NormalTok{\}}
\end{Highlighting}
\end{Shaded}

\section{Choose the mode of the
document!}\label{choose-the-mode-of-the-document}

\begin{verbatim}
## Data transformation performed for: none
\end{verbatim}

\begin{verbatim}
## Data integration perfomed for: PeakArea
\end{verbatim}

\section{Input files}\label{input-files}

\begin{verbatim}
## MTXQCparams.csv imported!
\end{verbatim}

\begin{verbatim}
## Maui_params.csv imported.
\end{verbatim}

\begin{verbatim}
## Number of modified peak areas: 100
\end{verbatim}

\begin{verbatim}
## `stat_bin()` using `bins = 30`. Pick better value with `binwidth`.
\end{verbatim}

\begin{center}\includegraphics{test/figure/MTXQCp3-evalQuant-1} \end{center}

\begin{verbatim}
## Manual validated peak areas have been merged original data and saved in: input/quant/MassAreasMatrix_ManVal.csv
\end{verbatim}

\begin{center}\includegraphics{test/figure/MTXQCp3-evalQuant-2} \end{center}
\newpage
\singlespacing 
\end{document}